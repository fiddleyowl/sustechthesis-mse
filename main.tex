% !Mode:: "TeX:UTF-8"
% !TEX program  = xelatex
% !BIB program  = biber
\documentclass[AutoFakeBold,AutoFakeSlant,language=chinese,degree=bachelor]{sustechthesis}
% 1. AutoFakeBold 与 AutoFakeSlant 为伪粗与伪斜,如果本机上有相应粗体与斜体字体,请使用 xeCJK 宏包进行设置,例如:
%   \setCJKmainfont[
%     UprightFont = * Light,
%     BoldFont = * Bold,
%     ItalicFont = Kaiti SC,
%     BoldItalicFont = Kaiti SC Bold,
%   ]{Songti SC}
%
% 2. language=chinese 基于为 ctexart 文类提供的中文排版方案修改,如果使用英文进行论文创作,请使用 language=english 选项。
%
% 3. degree=bachelor 为 sustechthesis 文类提供的本科生毕业论文模板,其他可选项为 master 与 doctor,但是均未实现,如果您对此有兴趣,欢迎 PR。
%
% 4. sustechthesis.cls 文类主要参考自去年完成使命的 sustechthesis.tex,在这一年的时间,作者的 TeX 风格与常用宏包发生许多变化,因为之前的思想为仅提供必要的格式修改相关代码,所以转换为文类形式所进行的修改较少,而近期的风格与常用宏包均体现在以下的例子文件中。
%
% 5. 示例文件均放置于相应目录的 examples 文件夹下,构建自己论文时可暂时保留,用以检索接口与使用方法。
%
% 6. 英文目录需要居中可以使用:\renewcommand{\contentsname}{\centerline{Content}}
%
% 7. LaTeX 中公式编号括号样式及章节关联的方法:https://liam.page/2013/08/23/LaTeX-Formula-Number/

% !Mode:: "TeX:UTF-8"
% !TEX program  = xelatex

% 数学符号与环境
\usepackage{amsmath,amssymb}
  \newcommand{\dd}{\mathrm{d}}
  \newcommand{\RR}{\mathbb{R}}
% 参考文献
\usepackage[style=gb7714-2015,gbpunctin=false]{biblatex}
  \addbibresource{ref.bib}
\AtEveryBibitem{\clearfield{doi}\clearfield{url}\clearfield{issn}\clearfield{urlyear}\clearfield{urlmonth}\clearfield{urlday}}
% 无意义文本
\usepackage{zhlipsum,lipsum}
% 列表环境设置
\usepackage{enumitem}
% 浮动题不越过 \section
\usepackage[section]{placeins}
% 超链接
\usepackage{hyperref}
% 图片,子图,浮动题设置
\usepackage{graphicx,subcaption,float}
% 抄录环境设置,更多有趣例子请命令行输入 `texdoc tcolorbox`
\usepackage{tcolorbox}
  \tcbuselibrary{xparse}
  \DeclareTotalTCBox{\verbbox}{ O{green} v !O{} }%
    {fontupper=\ttfamily,nobeforeafter,tcbox raise base,%
    arc=0pt,outer arc=0pt,top=0pt,bottom=0pt,left=0mm,%
    right=0mm,leftrule=0pt,rightrule=0pt,toprule=0.3mm,%
    bottomrule=0.3mm,boxsep=0.5mm,bottomrule=0.3mm,boxsep=0.5mm,%
    colback=#1!10!white,colframe=#1!50!black,#3}{#2}%
\tcbuselibrary{listings,breakable}
  \newtcbinputlisting{\Python}[2]{
    listing options={language=Python,numbers=left,numberstyle=\tiny,
      breaklines,commentstyle=\color{white!50!black}\textit},
    title=\texttt{#1},listing only,breakable,
    left=6mm,right=6mm,top=2mm,bottom=2mm,listing file={#2}}
% 三线表支持
\usepackage{booktabs}

% LaTeX logo
\usepackage{hologo}
 % 导言区
% !Mode:: "TeX:UTF-8"
% !TEX program  = xelatex
\设置信息{
    % 键 = {{中文值}, {英文值}},
    分类号 = {{}, {}},
    编号 = {{}, {}},
    UDC = {{}, {}},
    密级 = {{}, {}},
    % 仅题目(不含副标题)、系别、专业,支持手动 \\ 换行,不支持自动换行。
    题目 = {{南方科技大学材料科学与工程系\\ 毕业论文 \hologo{LaTeX} 形式 v\version}, {Graduation Thesis Template\\MSE \hologo{LaTeX} Format v\version}},
    % 如无需副标题,删除值内容即可,不可删除键定义。
    副标题 = {{}, {}}, % 材料系模板中无副标题
    姓名 = {{姓\hspace*{1em}名}, {First Last}}, % 若只有两个字,可使用 \hspace*{1em} 添加一个空格
    学号 = {{12010000}, {12010000}},
    系别 = {{材料科学与工程系}, {Department of Materials Science \\ and Engineering}},
    专业 = {{材料科学与工程}, {Materials Science and Engineering}},
    指导老师 = {{教授姓名}, {First Last}},
    时间 = {{2024年6月10日}, {June 10, 2024}},
    职称 = {{教授}, {Professor}},
}
 % 论文信息
\begin{document}

\中文标题页\英文标题页

\中文诚信承诺书
\英文诚信承诺书

\前序格式化
\摘要标题
% !Mode:: "TeX:UTF-8"
% !TEX program  = xelatex
\begin{中文摘要}{\LaTeX ;接口}
  \addcontentsline{toc}{section}{摘\hspace*{1em}要} % 在目录里添加摘要
  笔者见到的毕业论文模板,大多是以文类的形式,少部分以宏包的形式,并且在模板中大多掺杂着各式各样的例子(除了维护频率高的模板),导致模板文件使用了大部分与形式格式不相关的内容,代码量巨大文档欠缺且不容易修改,出现问题需要查看宏包或者文类的源代码。于是,秉着仅提供实现最基本要求的理念,重构了之前所写的 \TeX\ 形式。由于第二年使用该模板,所以设计出的模板接口不能保证以后不发生重大变动,一切以文档为主。毕竟学校在发展初期,各类文件都在日渐完善,前几年时,学校标志及名称还发生变化,同时毕业论文的样式也发生了重大变化。但是可以保证的是,模板提供的接口均为中文形式\footnote{使用 \hologo{XeLaTeX} 特性,一方面增加辨识度,另一方面不拘泥于英文命名的规则。当然此举也有些许弊端,在此就不过多展开。},并且至少更新到 2021 年,也就是笔者毕业。模板这种东西不能保证一劳永逸,一方面学校的标准制度都在发生着改变,另一方面 \hologo{LaTeX} 的宏包也在发生着改变,早先流行的宏包可能几年后就被“淘汰”掉。因此,您的使用与反馈是我不断更新的动力,希望各位不吝赐教。
\end{中文摘要}

\begin{英文摘要}{LaTeX, Interface}
  \addcontentsline{toc}{section}{ABSTRACT} % 在目录里添加摘要
  \lipsum[1]
\end{英文摘要}
 % 论文摘要

\目录\clearpage % 目录及换页

\正文格式化
% !Mode:: "TeX:UTF-8"
% !TEX program  = xelatex
\section{免责声明} % 如为引言,使用 \section{引\hspace*{1em}言}
\begin{enumerate}
    \item 本模板的发布遵守 \LaTeX\ Project Public License,使用前请认真阅读协议内容。
    \item 南方科技大学教学工作部只提供毕业论文写作指南,不提供官方模板,也不会授权第三方模板为官方模板,所以此模板仅为写作指南的参考实现,不保证格式审查老师不提意见. 任何由于使用本模板而引起的论文格式审查问题均与本模板作者无关。
    \item 任何个人或组织以本模板为基础进行修改,扩展而生成的新的专用模板,请严格遵守 \LaTeX\ Project Public License 协议。由于违犯协议而引起的任何纠纷争端均与本模板作者无关。
\end{enumerate}

% !Mode:: "TeX:UTF-8"
% !TEX program  = xelatex
\section{文类接口}
文类的接口的命名均为汉字,意思为字面意思,如有疑问,欢迎在 GitHub 提出 \href{https://github.com/Iydon/sustechthesis/issues}{Issues}。

\subsection{汉化字号接口}
本接口主要使用 \texttt{ctex} 宏包。

\verbbox{\初号},\verbbox{\小初},\verbbox{\一号},\verbbox{\小一},\verbbox{\二号},\verbbox{\小二},\verbbox{\三号},\verbbox{\小三},\verbbox{\四号},\verbbox{\小四},\verbbox{\五号},\verbbox{\小五},\verbbox{\六号},\verbbox{\小六},\verbbox{\七号},\verbbox{\八号}。


\subsection{汉化字体接口}
可能本机上部分字体不存在,导致部分字体无法使用。

\verbbox{\宋体},\verbbox{\黑体},\verbbox{\仿宋},\verbbox{\楷书},\verbbox{\隶书},\verbbox{\幼圆},\verbbox{\雅黑},\verbbox{\苹方}。


\subsection{字体效果接口}

建议在正文时使用 \verb|\textbf{}|,\verb|\textit{}| 调用\textbf{粗体}与\textit{斜体}。

It is recommended to use \verb|\textbf{}|,\verb|\textit{}| to call \textbf{Bold} and \textit{ItalicFont}.

\verbbox{\粗体},\verbbox{\斜体}。


\subsection{格式相关接口}
\subsubsection{命令}
例子请参考前文,在写论文初期,可以注释掉标题页等不必要信息,以加快编译速度。

\verbbox{\设置信息},\verbbox{\目录},\verbbox{\下划线},\verbbox{\中文标题页},\verbbox{\英文标题页},\verbbox{\中文诚信承诺书},\verbbox{\英文诚信承诺书},\verbbox{\摘要标题},\verbbox{\参考文献},\verbbox{\附录},\verbbox{\致谢}。

\subsubsection{环境}
摘要环境均需一个参数,为关键词:\verb|\begin{}{}...\end{}|。

\verbbox{中文摘要},\verbbox{英文摘要}。

% !Mode:: "TeX:UTF-8"
% !TEX program  = xelatex

\section{一些样例}

\subsection{参考文献}

参考文献一般使用\verbbox{\cite{<key>}}命令,效果如是\cite{Nicholas1998Handbook},引用作者使用 \\ \verbbox{\citeauthor{<key>}},效果如是“\citeauthor{goossens1994latex}”。

\subsection{表格}

表格与图片可以直接通过\verbbox{\ref{<key>}}来引用,例如表 \ref{table2}、图 \ref{F:test-a}、图 \ref{F:test-b-sub-b}。材料科学与工程系模板中表格为双倍行距,为不影响其余文字的正常行距,在 \\ \verbbox{\begin{table}}后使用\verbbox{\renewcommand{\arraystretch}{1.5}}使得仅在表格内部使用双倍行距。

\begin{table}[htb]
% h-here,t-top,b-bottom,优先级依次下降
    % 居中,模版已设定表格浮动体居中
    \renewcommand{\arraystretch}{1.5} % 双倍行距
    \centering
    \caption{表格的标题应该放在上方}
    \label{table}
    \begin{tabular}{lc} % 三线表不能有竖线,l-left,c-center,r-right
        \toprule
        %三线表-top 线
        Example & Result \\
        \midrule
        %三线表-middle 线
        Example1          & 0.25 \\
        Example2          & 0.36 \\
        \bottomrule
        %三线表-底线
    \end{tabular}
\end{table}

\begin{table}[htb]
    \renewcommand{\arraystretch}{1.5} % 双倍行距
    \centering
    \caption{带表注的表格的标题}
    \label{table2}
    \begin{threeparttable}
        \setlength{\tabcolsep}{0.6cm}{ % 调节表格长度
                \begin{tabular}{lc} % 三线表不能有竖线,l-left,c-center,r-right
                    \toprule
                    %三线表-top 线
                    Example & Result \\
                    \midrule
                    %三线表-middle 线
                    Example1          & 0.25\tnote{1} \\
                    Example2          & 0.36 \\
                    \bottomrule
                    %三线表-底线
                \end{tabular}
        }
        \begin{tablenotes}
            \item[1] 数据来源:南方科技大学 \LaTeX 模版 % 增加表格数据来源注释
        \end{tablenotes}
    \end{threeparttable}
\end{table}

\subsection{图片}

在材料科学与工程系的模板中,图片需要和论文组图一样在子图的左上角标序,并在全图的说明文本内说明每张子图的含义。在子图中使用\verbbox{\caption}命令时,需要该命令放置在\verbbox{\includegraphics}命令之前。

图 \ref{F:test-a} 是单张示例图片,图 \ref{F:test-b} 是两张图片的组图示例,图 \ref{F:test-c} 是四张图片的组图示例。更多图片的排版方式可参照这些示例。

\clearpage

\begin{figure}[htb]
    \centering
    \includegraphics[width=.5\textwidth]{example-image-a}
    \caption{自带测试图片---Test image}\label{F:test-a}
    % 图片的标题应该在下方
\end{figure}

\begin{figure}[htb]
    \centering
    \begin{subfigure}[t]{.45\linewidth}
        \centering
        \caption{}\label{F:test-b-sub-a}
        \includegraphics[width=1\textwidth]{example-image-a}
    \end{subfigure}
    \hfill % 两张图片之间的间隔,确保第二张图片的标题不会挨着第一张图片的内容
    \begin{subfigure}[t]{.45\linewidth}
        \centering
        \caption{}\label{F:test-b-sub-b}
        \includegraphics[width=1\textwidth]{example-image-b}
    \end{subfigure}
    \caption{自带测试图片---Test image,其中(a)为测试图片A,(b)为测试图片B。}\label{F:test-b}
\end{figure}

如果一页恰好没有文字,只有图片,\LaTeX 在默认状态下会将这些图片居中放置而不是从最上方开始。如果需要将图片放置在最上方,可以在\verbbox{\clearpage}手动换页后,尝试\verbbox{\begin{figure}[!ht]}命令。

\clearpage

\begin{figure}[!ht]
    \centering
    \begin{subfigure}[t]{.45\linewidth}
        \centering
        \caption{}\label{F:test-c-sub-a}
        \includegraphics[width=1\textwidth]{example-image-a}
    \end{subfigure}
    \hfill % 两张图片之间的间隔,确保第二张图片的标题不会挨着第一张图片的内容
    \begin{subfigure}[t]{.45\linewidth}
        \centering
        \caption{}\label{F:test-c-sub-b}
        \includegraphics[width=1\textwidth]{example-image-b}
    \end{subfigure} % 换行位置
    \begin{subfigure}[t]{.45\linewidth}
        \centering
        \caption{}\label{F:test-c-sub-c}
        \includegraphics[width=1\textwidth]{example-image-c}
    \end{subfigure}
    \hfill
    \begin{subfigure}[t]{.45\linewidth}
        \centering
        \caption{}\label{F:test-c-sub-d}
        \includegraphics[width=1\textwidth]{example-image-a}
    \end{subfigure}
    \caption{自带测试图片---Test image,其中(a)为测试图片A,(b)为测试图片B,(c)为测试图片C,(d)也为测试图片A。}\label{F:test-c}
\end{figure}

\subsection{文字}

\subsubsection{花菁素类染料}

花菁素类染料是指以具有聚甲炔(Polymethine)共轭结构为发色团的染料。花菁素类染料通常具有一条聚甲炔链,两端与两个氮原子相连。两个氮原子分别属于两个芳杂环,例如吡咯、吡啶、吲哚、喹啉。最早用于活体近红外荧光成像的花菁素类染料是已经被美国食品和药物管理局(FDA)批准用于人体临床使用的吲哚菁绿(ICG)。

\subsubsection{Nonsteady-state Diffusion}

Most practical diffusion situations are nonsteady-state ones. That is, the diffusion flux and the concentration gradient at some particular point in a solid vary with time, with a net accumulation or depletion of the diffusing species resulting. This is illustrated in Figure x, which shows concentration profiles at three different diffusion times. Under conditions of nonsteady state, use of Equation \ref{eq:diffusion} is no longer convenient; instead, the partial differential equation
\begin{equation}
    \frac{\partial C}{\partial t} = \frac{\partial}{\partial x} (D \frac{\partial C}{\partial x})
    \label{eq:diffusion}
\end{equation}
known as \textbf{Fick's second law}, is used. If the diffusion coefficient is independent of composition (which should be verified for each particular diffusion situation), Equation \ref{eq:diffusion} simplifies to
\begin{equation}
    \frac{\partial C}{\partial t} = D \frac{\partial^2 C}{\partial x^2}
    \label{eq:diffusion2}
\end{equation}

\subsubsection{Determination of Phase Amounts}

Consider again the example shown in Figure x, in which at $\mathrm{1250 \,^\circ C}$ both $\alpha$ and liquid phases are present for a 35 wt\% Ni-65 wt\% Cu alloy. The problem is to compute the fraction of each of the $\alpha$ and liquid phases. The tie line has been constructed that was used for the determination of $\alpha$ and $L$ phase compositions. Let the overall alloy composition be located along the tie line and denoted as $C_0$, and mass fractions be represented by $W_L$ and $W_\alpha$ for the respective phases. From the lever rule, $W_L$ may be computed according to
\begin{align}
    W_L &= \frac{S}{R+S} \label{eq:lever1} \\
    &= \frac{C_\alpha-C_0}{C_\alpha-C_L} \label{eq:lever2}
\end{align}
% !Mode:: "TeX:UTF-8"
% !TEX program  = xelatex
\section{\LaTeX\ 入门}
请参考 \href{https://tex.readthedocs.io/zh_CN/latest/}{在线文档},包括学习资源及学习路径。欢迎在 GitHub 上提出 \href{https://github.com/Iydon/tex/issues}{Issues}。
\clearpage

\结论
  % !Mode:: "TeX:UTF-8"
% !TEX program  = xelatex

发展高性能 NIR-II 荧光分子染料对推动 NIR-II 能易于调控的特点引起了研究者们的兴趣。然而,目前将其应用于 NIR-II 荧光成像的报道较少,且对其分子结构与染料在水中的光学性能间的关系研究较少。

本工作设计并合成了一系列 A-D-A 型稠环受体分子,进一步用两亲性聚合物包裹制备了水溶性 NIR-II 纳米荧光染料,通过对受体单元上卤素原子取代基 的数量和类型、氰基引入进行调控,结合理论计算,研究A 单元结构与染料在水中光学性能(摩尔消光系数、吸收与发射波长、QY)间的关系。\clearpage
\参考文献
  \printbibliography[heading=none]\clearpage
\附录
  % !Mode:: "TeX:UTF-8"
% !TEX program  = xelatex
\subsection*{数据获取函数}\label{A:data}
\Python{utils.py}{code/examples/utils.py}
\clearpage
\致谢
  \input{sections/examples/thanks.tex}
\end{document}
